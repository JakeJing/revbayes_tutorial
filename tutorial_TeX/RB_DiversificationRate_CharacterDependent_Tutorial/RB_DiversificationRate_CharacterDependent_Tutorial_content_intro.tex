\section*{Estimating Character State-Dependent Speciation \& Extinction Rates}

\subsection*{Introduction}

This tutorial describes how to specify character state-dependent branching process models in \RevBayes.
Frequently referred to as state-dependent speciation and extinction (SSE) models,
these models are a birth-death process where the diversification rates are dependent on the state of an evolving character.
The original model of this type considered a binary character (a trait with two discrete state values; called BiSSE, \citealt{Maddison2007}).
Several variants have also been developed for other types of traits \citep{FitzJohn2010, Goldberg2011, Goldberg2012, MagnusonFord2012, FitzJohn2012, Beaulieu2016, Freyman2017}.

We will outline the theory behind this method, and then you will fit it to data using Markov chain Monte Carlo (MCMC).
\RevBayes is a powerful tool for SSE analyses. After working through this tutorial you
should be able to set up custom SSE models and use them to
infer character-dependent diversification rates and ancestral states.
% and stochastic character maps. % maybe coming later?
We also provide examples of how to plot the results using the \RevGadgets R package.

\subsection*{Contents}

The State-Dependent Speciation and Extinction tutorial contains several sections:

\begin{itemize}
\item Section \ref{sec:diversification_rate_overview}: Introduction to diversification rate estimation
\item Section \ref{sec:models}: Theory behind diversification rate models
\item Section \ref{sec:BiSSE_Theory}: Theory behind SSE models
\item Section \ref{sec:CDBDP}: Running a BiSSE/MuSSE analysis in \RevBayes
\item Section \ref{sec:HiSSE_Theory}: Running a HiSSE analysis in \RevBayes
\item Section \ref{sec:ClaSSE}: Running a ClaSSE analysis in \RevBayes
\end{itemize}

\subsection*{Requirements}

We assume that you have read and hopefully completed the following tutorials:
\begin{itemize}
\item \href{https://github.com/revbayes/revbayes_tutorial/raw/master/tutorial_TeX/RB_Getting_Started/RB_Getting_Started.pdf}{Getting started}
\item \href{https://github.com/revbayes/revbayes_tutorial/raw/master/tutorial_TeX/RB_Intro_Tutorial/RB_Intro_Tutorial.pdf}{Very Basic Introduction to \Rev}
\item \href{https://github.com/revbayes/revbayes_tutorial/raw/master/tutorial_TeX/RB_Rev_Tutorial/RB_Rev_Tutorial.pdf}{General Introduction to the \Rev syntax}
\item \href{https://github.com/revbayes/revbayes_tutorial/raw/master/tutorial_TeX/RB_MCMC_Archery_Tutorial/RB_MCMC_Archery_Tutorial.pdf}{General Introduction to MCMC using an archery example}
\item \href{https://github.com/revbayes/revbayes_tutorial/raw/master/tutorial_TeX/RB_MCMC_Binomial_Tutorial/RB_MCMC_Binomial_Tutorial.pdf}{General Introduction to MCMC using a coin-flipping example}
\item \href{https://github.com/revbayes/revbayes_tutorial/raw/master/tutorial_TeX/RB_DiversificationRate_Tutorial/RB_DiversificationRate_Tutorial.pdf}{Basic Diversification Rate Estimation}
\end{itemize}
Note that the \href{https://github.com/revbayes/revbayes_tutorial/raw/master/tutorial_TeX/RB_Intro_Tutorial/RB_Intro_Tutorial.pdf}{\Rev basics tutorial} introduces the basic syntax of \Rev but does not cover any phylogenetic models.
We tried to keep this tutorial very basic and introduce all the language concepts and theory on the way.
You may only need the \href{https://github.com/revbayes/revbayes_tutorial/raw/master/tutorial_TeX/RB_Rev_Tutorial/RB_Rev_Tutorial.pdf}{\Rev syntax tutorial} for a more in-depth discussion of concepts in \Rev.

%%%%%%%%%%%%%%
%%   DATA   %%
%%%%%%%%%%%%%%
\section*{Data and files} \label{sec:data_files}

We provide the data files which we will use in this tutorial.
You may want to use your own data instead.
In the \cl{data} folder, you will find the following files:
\begin{itemize}

    \item \href{https://github.com/revbayes/revbayes_tutorial/raw/master/RB_DiversificationRate_CharacterDependent_Tutorial/data/primates_tree.nex}{primates\_tree.nex}:
        Dated primate phylogeny including 233 out of 367 species.
        (This tree is from \citealt{MagnusonFord2012}, who took it from \citealt{Vos2006} and then randomly resolved the polytomies using the method of \citealt{Kuhn2011}.)

    % Not actually used in this tutorial.
    % \item \href{https://github.com/revbayes/revbayes_tutorial/raw/master/RB_DiversificationRate_CharacterDependent_Tutorial/data/primates_habitat.nex}{primates\_habitat.nex}:
    %     Habitat data (also from \citealt{MagnusonFord2012}).

    \item \href{https://github.com/revbayes/revbayes_tutorial/raw/master/RB_DiversificationRate_CharacterDependent_Tutorial/data/primates_morph.nex}{primates\_morph.nex}:
        A set of several discrete-valued characters. % from where?
        The characters are described in the file \href{https://github.com/revbayes/revbayes_tutorial/raw/master/RB_DiversificationRate_CharacterDependent_Tutorial/data/primates_morph_description.txt}{primates\_morph\_description.txt}.

    \item \href{https://github.com/revbayes/revbayes_tutorial/raw/master/RB_DiversificationRate_CharacterDependent_Tutorial/data/primates_biogeo.tre}{primates\_biogeo.tre}: 
        A dated phylogeny of the 23 primate species. % from M. Landis' biogeo DEC tutorial, not sure where he got it from...
    
    \item \href{https://github.com/revbayes/revbayes_tutorial/raw/master/RB_DiversificationRate_CharacterDependent_Tutorial/data/primates_biogeo.tsv}{primates\_biogeo.tsv}:
        Biogeographic range data for 23 primate species.

\end{itemize}

\impmark{Open the tree files \cl{primates\_tree.nex} and \cl{primates\_biogeo.tre} in FigTree.}

\impmark{Open the character data files \cl{primates\_morph.nex} and \cl{primates\_biogeo.tsv} in a text editor.}

\newpage

