
\section{Overview}

\RevBayes has as its central idea that any statistical model, for example a phylogenetic model, is composed of smaller parts that can be decomposed and put back together in a modular fashion \citep{Hoehna2016b}. 
This comes from considering (phylogenetic) models as \textit{probabilistic graphical models}, which lends flexibility and enhances the capabilities of the program. 
Users interact with \RevBayes via an interactive shell.
Users communicate commands using a language specifically designed for \RevBayes, called \Rev; an R-like language (complete with control statements, user-defined functions, and loops) that enables the user to build up (phylogenetic) models from simple parts (random variables, variable/parameter transformations, models, and constants of different sorts).
 
Here we assume that you have successfully installed \RevBayes.
If this isn't the case, then please consult our website on how to install \RevBayes.



\bigskip
\section{Getting Started}

For the examples outlined in each tutorial, we will use \RevBayes interactively by typing commands in the command-line console.
For the exercises you can either use \RevBayes interactively or run an entire script.
Execute the \RevBayes binary. If this program is in your path, then you can simply type in your Unix terminal:

\exs{\cl{\$ rb}}

When you execute the program, you will see a brief program information, including the current version number. 
Remember that more information can be obtained from \href{www.RevBayes.com}{www.RevBayes.com}.
When you execute the program with an additional filename, \EG
\exs{\cl{\$ rb my\_analysis.Rev}}
then \RevBayes will run all commands specified in your file.

You may want to run \RevBayes in parallel using multiple processes.
This can be done by starting \RevBayes with
\exs{\cl{\$ mpirun -np 4 rb-mpi}}
which starts 4 processes of \RevBayes.
You may want to change the number of processes depending on your available hardware.

The format of the exercises uses \colorbox{shadecolor}{\tt blue shaded boxes} to delineate code examples that you should type into \RevBayes. 
For example, after opening the \RevBayes program, you can load your data file:
{\tt \begin{snugshade*}
\begin{lstlisting}
data <- readDiscreteCharacterData("data/primates_cytb.nex")
\end{lstlisting}
\end{snugshade*}}

%Multi-line entries, particularly loops, will often be displayed in boxes.
The \cl{RevBayes >} prompt that you see in your terminal is omitted so that the examples can be copied and pasted wholly. 
This is especially useful for larger command blocks, particularly loops, which will often be displayed in boxes
{\tt \begin{snugshade*}
\begin{lstlisting}
for( i in 1:12 ){
  x[i] ~ dnExponential(1.0)
}
\end{lstlisting}
\end{snugshade*}}

The various \RevBayes commands and syntax within the text are specified using \cl{typewriter text}. 

Most tutorials also includes hyperlinks: bibliographic citations are {\textcolor{citescol}{medium sea green}} and link to the full citation in the references, external URLs are {\textcolor{urlscol}{cerulean}}, and internal references to figures and equations are {\textcolor{linkscol}{forest green}}.

The various exercises in this tutorial take you through the steps required to perform phylogenetic analyses of the example datasets. 
In addition, we have provided the \Rev scripts and output files for some exercises so you can verify your results.
(Note that since the MCMC runs you perform will start from different random seeds, the output files resulting from your analyses \textit{will not} be identical to the ones we provide you.)


\bigskip
\section{Probabilistic Graphical Models}

\RevBayes uses \textit{probabilistic graphical models} for model specification, visualization, and implementation \citep{Hoehna2014b}. 
Graphical models are frequently used in machine learning and statistics to conceptually represent the conditional dependence structure of complex statistical models with many parameters \citep{Gilks1994,Lunn2000,Jordan2004,Koller2009,Lunn2009}. 
The graphical model framework allows for flexible model specification and implementation and reduces redundant code. 
This framework provides a set of symbols for depicting a \href{http://en.wikipedia.org/wiki/Directed_acyclic_graph}{\textit{directed acyclic graph}} (DAG). 
\citet{Hoehna2014b} described the use of probabilistic graphical models for phylogenetics. 
The different nodes and components of a phylogenetic graphical model are shown in Figure \ref{gmnotation} \citep[Fig. 1 from][]{Hoehna2014b}. 
\begin{figure}[h!]
\centering
\fbox{\includegraphics[width=1.8in,angle=0]{\ResourcePath figures/GM_notation_figure.eps}}
\caption{\small The symbols for a visual representation of a graphical model. 
a) Solid squares represent constant nodes, which specify fixed-valued variables. 
b) Stochastic nodes are represented by solid circles. 
These variables correspond to random variables and may depend on other variables. 
c) Deterministic nodes (dotted circles) indicate variables that are determined by a specific function applied to another variable. 
They can be thought of as variable transformations. 
d) Observed states are placed in clamped stochastic nodes, represented by gray-shaded circles. e) Replication over a set of variables is indicated by enclosing the replicated nodes in a plate (dashed rectangle). 
[Partially reproduced from Fig.~1 in \citet{Hoehna2014b}.]
}
\label{gmnotation}
\end{figure}

To represent the DAG, nodes are connected with arrows indicating dependency. 
A simple, albeit abstract, graphical model is shown in Figure \ref{simpleGM}. 
In this model, we observe a set of states for parameter $x$. 
We assume that the values of $x$ are samples from a lognormal distribution with a location parameter (log mean) $\mu$ and a standard deviation $\sigma$. 
It is more straightforward to model our uncertainty in the expectation of a lognormal distribution, rather than $\mu$, thus we place a gamma distribution on the mean $M$. 
This gamma hyperprior has two parameters that we specify with fixed values (constant nodes): the shape $\alpha$ and rate $\beta$. 
The variable $M$ is a stochastic node with this prior density.
The standard deviation, $\sigma$, is also a stochastic node with an exponential prior density with rate parameter $\lambda$.
For any value of $M$ and any value of $\sigma$ we can compute the deterministic variable $\mu$ using the formula $\mu = \ln(M) - \frac{\sigma^2}{2}$. 
This formula is known from using simple algebra on the equation for the mean of any \href{http://en.wikipedia.org/wiki/Log-normal_distribution}{lognormal distribution}.
With this model structure, we can then calculate the probability of the data conditional on the model and parameter values (the likelihood): 
$\mathbb{P}(\boldsymbol{x} \mid \mu, \sigma)$. 
Next we can get the posterior probability using Bayes' theorem:
$$\mathbb{P}(M,\sigma \mid \boldsymbol{x}, \alpha, \beta, \lambda) = \frac{\mathbb{P}(\boldsymbol{x} \mid \mu, \sigma) \mathbb{P}(M \mid \alpha,\beta) \mathbb{P}(\sigma \mid \lambda)}{\mathbb{P}(\boldsymbol{x})}.$$
\begin{figure}[h!]
\centering
\fbox{\includegraphics[width=2.5in,angle=0]{\ResourcePath figures/simple_GM.eps}}
\caption{\small Graphical model representation of a simple lognormal model. A total of $N$ observations of variable $x$ are observed and occupy a clamped node. 
This parameter is log-normally distributed with parameters $\mu$ and $\sigma$ (log mean and standard deviation, respectively). 
The parameter $\mu$ is a deterministic node that is calculated from the stochastic nodes $M$ (the mean of the distribution) and $\sigma$. 
Dotted arrows indicate deterministic functions and are used to connect deterministic nodes to their parent variables. 
A gamma distribution is applied as a hyperprior on $M$ with constant nodes for the shape $\alpha$ and rate $\beta$.
The stochastic variable $\sigma$ is exponentially distributed with fixed value for the rate $\lambda$.
}
\label{simpleGM}
\end{figure}



\bigskip
\section{\Rev: The \RevBayes Language}

In \RevBayes models and analyses are specified using an interpreted language called \Rev. 
\Rev bears similarities to the compiled language in WinBUGS and the interpreted \R language. 
Setting up and executing a statistical analysis in \RevBayes requires the user to specify all of the parameters of their model and the type of analysis (\EG an MCMC run). 
By using an interpreted language, \RevBayes enables the practitioner to build complex, hierarchical models and to check the current states of variables while building the model. 
This will be very useful in the beginning.
Later on you, when you run very complex analyses, you may want to write \Rev-scripts.

Differently to \R and BUGS, \Rev is a strongly but implicitly typed language.
It is implicitly typed, and thus similar to Python, because you do not need to provide the type of a variable (which you need to in languages such as C++ and Java).
We do implicit typing to help users who do not know about the actual types of the variables.
However, strongly typed means that every variable has a type and arguments of functions need to match the required types.
The strong type requirements ensures that you build meaningful model graphs. 
For example, the variance parameter of a normal distribution needs to be a positive number, and thus you can only use variables that are positive real numbers.
\RevBayes does automatic type conversion.

\bigskip
\subsection{Specifying Models}

\begin{table}[h!]
\centering
\caption{\Rev assignment operators, clamp function, and plate/loop syntax.}\label{operatorTable}
\begin{tabular}{@{\extracolsep{\fill}}l  c r }
\hline
\multicolumn{1}{l}{\textbf{Operator}} & \multicolumn{1}{c}{ } & \multicolumn{1}{r}{\textbf{Variable}}  \\ 
\hline
\cl{<-} & \hspace{10mm} &  constant variable\\
\cl{\rbdn} & \hspace{10mm} &  stochastic variable\\
\cl{:=} & \hspace{10mm} &  deterministic variable\\
\cl{node.clamp(data)} & \hspace{10mm} &  clamped variable\\
\cl{=} & \hspace{10mm} &  inference (\IE non-model) variable\\
\cl{for(i in 1:N)\{...\}} & \hspace{10mm} &  plate\\
\hline
\end{tabular}
\end{table}

The variables/parameters of a statistical model are created using different operators in \Rev (Table \ref{operatorTable}). 
In Figure \ref{revgmexample}, the \Rev syntax for creating the model in Figure \ref{simpleGM} is provided.
Because \Rev is an interpreted language, it is important to consider the order in which you specify your variables (\CF BUGS where the order is not important). 
Thus, typically the first variables that are instantiated are \emph{constant variables}. 
Constant variables require you to assign a fixed value using the \cl{<-} operator. 
Stochastic variables are initialized using the \cl{\rbdn} operator followed by the constructor function for a distribution. 
In \Rev, the naming convention for distributions is \cl{dn*}, where \cl{*} is a wildcard representing the name of the distribution. 
Each distribution function requires hyperparameters passed in as arguments.
This is effectively linking nodes using arrows in the graphical model.
The following code snippet creates a stochastic variable called \cl{M} which is assigned a gamma-distributed hyperprior, with shape \cl{alpha} and rate \cl{beta}:
{\tt \begin{snugshade*}
\begin{lstlisting}
alpha <- 2.0
beta <- 4.0
M ~ dnGamma(alpha, beta)
\end{lstlisting}
\end{snugshade*}}

The flexibility gained from the graphical model framework and the interpreted language allows you to easily change a model by swapping components. 
For example, if you decide that a bimodal lognormal distribution is a better representation of your uncertainty in \cl{M}, then you can simply change the distribution associated with \cl{M} (after initializing the bimodal lognormal hyperparameters):
{\tt \begin{snugshade*}
\begin{lstlisting}
mean_1 <- 0.5
mean_2 <- 2.0
sd_1 <- 1.0
sd_2 <- 1.0
weight <- 0.5
M ~ dnBimodalLognormal(mean_1, mean_2, sd_1, sd_2, weight)
\end{lstlisting}
\end{snugshade*}}

\Rev does allow you to specify constant-variable values in the distribution constructor function; therefore this also works:
{\tt \begin{snugshade*}
\begin{lstlisting}
M ~ dnBimodalLognormal(0.5, 2.0, 1.0, 1.0, 0.5)
\end{lstlisting}
\end{snugshade*}}
Both ways to specify priors are equivalent.
The only difference is that one code may be more readable than the other.



\begin{figure}[h!]
\centering
\fbox{\includegraphics[width=5in,angle=0]{\ResourcePath figures/simple_GM_rev.eps}}
\caption{\small Specifying a model with \Rev. 
The graphical model of the observed parameter $x$ is shown on the left. 
In this example, $x$ is log-normally distributed with a location parameter of $\mu$ and a standard deviation of $\sigma$, thus $x \sim \mbox{Lognormal}(\mu, \sigma)$. 
The expected value of $x$ (or mean) is equal to $M$: $\mathbb{E}(x) = M$. 
In this model, $M$ and $\sigma$ are random variables and each are assigned hyperpriors. 
We assume that the mean is drawn from a gamma distribution with shape parameter $\alpha$ and rate parameter $\beta$: $M \sim \mbox{Gamma}(\alpha, \beta)$. 
The standard deviation of the lognormal distribution is assigned an exponential hyperprior with rate $\lambda$: $\sigma \sim \mbox{Exponential}(\lambda)$. 
Since we are conditioning our model on the \emph{expectation}, we must compute the location parameter ($\mu$) to 
calculate the probability of our model. 
Thus, $\mu$ is a deterministic node that is the result of the function$^*$ executed on $M$ and $\sigma$: $\mu = \ln(M) - \frac{\sigma^2}{2}$. 
Since we observe values of $x$, we \emph{clamp} this node.
}
\label{revgmexample}
\end{figure}

Deterministic variables are parameter transformations and initialized using the \cl{:=} operator followed by the function or formula for calculating the value. 
Previously we created a variable for the expectation of the lognormal distribution.
Now, if you have an exponentially distributed stochastic variable $\sigma$, you can create a deterministic variable for the mean $\mu$:
{\tt \begin{snugshade*}
\begin{lstlisting}
lambda <- 1.0
sigma ~ dnExponential(lambda)
mu := ln(M) - (sigma^2)/2.0
\end{lstlisting}
\end{snugshade*}}

Replication over lists of variables as a plate object is specified using \cl{for} loops. 
A for-loop is an iterator statement that performs a function a given number of times. 
In \Rev you can use this syntax to create a vector of 7 stochastic variables, each drawn from a lognormal distribution:
{\tt \begin{snugshade*}
\begin{lstlisting}
for( i in 1:7 ) {
  x[i] ~ dnLognormal(mu, sigma)
}
\end{lstlisting}
\end{snugshade*}}
The \cl{for} loop executes the statement \cl{x[i] $\sim$ dnLognormal(mu, sigma)} for different values of $i$ repeatedly, where $i$ takes the values 1 to 7.
Thus, we created a vector $x$ of seven variables, each being independent and identically distributed (\IID).

A clamped node/variable has observed data attached to it. 
Thus, you must first read in or input the data, then clamp it to a stochastic variable. 
In Figure \ref{revgmexample} the observations are assigned and clamped to the stochastic variables.
If we observed 7 values for \cl{x} we would create 7 clamped variables:
{\tt \begin{snugshade*}
\begin{lstlisting}
observations <- [0.20, 0.21, 0.03, 0.40, 0.65, 0.87, 0.22]
N <- observations.size()
for( i in 1:N ){
  x[i].clamp(observations[i])
}
\end{lstlisting}
\end{snugshade*}}
You may notice that the value of $x$ has now changed and is equal to the observations.



\section{Getting help in \RevBayes}

\RevBayes provides an elaborate help system. 
Most of the help is found online on our website \\
http://www.RevBayes.com.
Within \RevBayes you can display the help for a function, distribution or any other type using the \cl{?} symbol followed by the command you want help for:
{\tt \begin{snugshade*}
\begin{lstlisting}
?dnNorm
?mcmc
?mcmc.run
\end{lstlisting}
\end{snugshade*}}

Additionally, \RevBayes will print the correct usage of a function if you only type in its name and hit return:
{\tt \small \begin{snugshade*}
\begin{lstlisting}
mcmc
|*   MCMC function (Model model, Monitor[] monitors, Move[] moves, String moveschedule = "sequential" | "random" | "single", Natural nruns)
\end{lstlisting}
\end{snugshade*}}

If you typed in \cl{?dnNorm} and you didn't see the help but got instead an error message then you have most likely an incorrect path variable to the help directory.
You can check the current path to help directory by
{\tt \small \begin{snugshade*}
\begin{lstlisting}
getOption("helpdir")
|*   "/Users/hoehna/Software-Development/revbayes-development/help"
\end{lstlisting}
\end{snugshade*}}
Check where the help files on your system are and then set the correct path
{\tt \small \begin{snugshade*}
\begin{lstlisting}
setOption("helpdir", "/Users/hoehna/Software-Development/revbayes-development/help")
\end{lstlisting}
\end{snugshade*}}

\bigskip
\subsection{RevBayes Users' Forum}

An email list has been created for users of \RevBayes to discuss \RevBayes-related topics, including: \RevBayes installation and use, scripting and programming, phylogenetics, population genetics, models of evolution, graphical models, etc. The forum is hosted by Google Groups:

\exs{\href{http://bit.ly/107aW2R}{revbayes-users}}


\bibliographystyle{sysbio}
\bibliography{\GlobalResourcePath refs}

